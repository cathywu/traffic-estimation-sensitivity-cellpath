\documentclass[twocolumn,10pt]{article}
\usepackage{amsmath}

\begin{document}

\subsection{System model}
This work models the assignment of individual vehicles to cell towers with an effective received signal strength (eRSS) based decision process.  This eRSS between a vehicle $v$ and a cell tower $t$ is calculated from the received power by introducing tower handoff dynamics as noise penalties;
each vehicle gets assigned to the cell tower with the highest eRSS value at each time point in the simulation.
All towers are assumed to output with the same transmit power, so eRSS can be calculated in dB as: \begin{eqnarray}
  eRSS_{v-t} & = & \left( \frac{P_{RX_v}}{P_{TX_t}} \right)_{dB} 
                   + P_{handoff_{v-t}} \\
  & = & P_{d_{v-t}} +P_{int} + P_{hyst_{v-t}} + P_{load_t}.
\end{eqnarray}

In this equation, the actual received signal strength is composed of two parts, $P_d$ and $P_{int}$.  The attenuation due to path loss is given by: \begin{equation}
  P_{d_{v-t}} = -10 \gamma \log_{10} d_{v-t},
\end{equation}
where $d_{v-t}$ is the distance between the vehicle and the cell tower, and $\gamma$ is the path loss exponent.  The additional attenuation due to interference, multipath effects, and other RF noise is wrapped up into a uniformly distributed random penalty: \begin{equation}
  P_{int} = -\alpha \cdot \mathrm{Unif}(0,1).
\end{equation}

The dynamics of the tower handoff, $P_{handoff}$, are modelled using two parameters.  $P_{hyst}$ represents an amount of hysteresis in the system by boosting the eRSS of the currently associated tower, biasing individual vehicles against switching towers: \begin{equation}
  P_{hyst_{v-t}} = \begin{cases}
    \beta & \mathrm{if~} v \mathrm{~is~already~connected~to~} t, \\
    0 & \mathrm{otherwise.}
  \end{cases}
\end{equation}
$P_{load}$ represents the load balancing efforts to distribute usage across the cell network, adding an eRSS penalty to heavily loaded towers: \begin{equation}
  P_{load_t} = -\xi \cdot N_{t},
\end{equation}
where $N_t$ is the number of other vehicles already connected to tower $t$.  The four parameters $\{\gamma, \alpha, \beta, \xi\}$ can be scaled by the first to get our final parameter vector $\phi = \{a, b, c\} = \{\alpha/\gamma, \beta/\gamma, \xi/\gamma\}$ over which we conduct our sensitivity analysis, evaluating the relative impacts of hysteresis and load balancing in the presence of RF noise on route flow estimation.

\subsection{Limitations}
  
In this work, we only consider hard handoffs, wherein each vehicle is connected to exactly one cell tower at all times.  The load balancing is represented by a linear penalty on tower utilization, which ignores nonlinear effects such as maximum tower capacity.  The units are arbitrarily scaled, and so require manual tuning to determine appropriate relative weights.

\end{document}
