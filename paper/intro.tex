%!TEX root = main.tex

\section{Introduction}
Route flow estimation on road networks is a critical problem for mobility CPS, as real-time prediction and control applications require the state of the road network \cite{Faouzi2011,Mathew2014}. Previous work has demonstrated the promise of computing an estimate from area-based sensors such as cellular towers, using a projected first-order method that scales well to large city-scale networks \cite{Wu2015}. Importantly, the work's introduction of \textit{cellpaths} -- cell tower connection sequences -- helps to cope with the underdetermined-ness of the problem. However, previous analysis has been based off of noiseless synthesized data, which makes the strong but common assumption that vehicles are connected to the closest cell tower throughout their trajectory (i.e. Voronoi tessellations of cell tower regions) \cite{Voronoi1908}. The real world is seldom so kind \cite{Caceres2013}.

In this work we analyze the sensitivity of the flow estimation procedure to structured noise. In particular, we consider the effect of a handoff model for within the cellular network, which includes hysteresis margins for handoffs and cell tower load balancing, in the presence of RF interference.
